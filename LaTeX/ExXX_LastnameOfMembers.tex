\documentclass[11pt]{scrartcl}
%% Encoding, font & language setup:
\usepackage[noTeX]{mmap}	% For UTF-8 text marking
\usepackage[nomath]{lmodern}	% Ensures modern latin font family; must be loaded before fontenc; [nomath] parameter for compatibility with mmap package
\usepackage[utf8]{inputenc}		% Another nice parameter: [latin9]
\usepackage[EU1,T1]{fontenc}	% Sets the right font encoding
\usepackage[ngerman]{babel}
\usepackage[style=german]{csquotes}		% Handles quotation

%% Page formatting:
\usepackage[
    a4paper,
    lmargin={1.5cm},
    rmargin={1.5cm},
    tmargin={1.5cm},
    bmargin = {1.5cm},
    headheight={1.5cm},
    includeheadfoot
]{geometry}		% Page size
\usepackage{fancyhdr}	% Header package

%% Considered useful:
\usepackage{float}		% Sets positioning of sections witch [H] to place of occurrence instead of the next (sub)section with the [H] parameter
\usepackage{changepage}		% Adjust left/ right side margins with \begin{adjustwidth}{24pt} ... \end{adjustwidth}
\usepackage[backend=bibtex,style=verbose-trad2,autocite=footnote]{biblatex}		% For compatibility with VSCode pdflatex -> bibtex -> pdflatex x 2

%% Basic packages:
\usepackage[unicode]{hyperref}
\usepackage{bookmark}
\usepackage[fixpdftex,hyperref,table,dvipsnames]{xcolor}	% Everything color-related
\usepackage{graphicx}		% For multiple figures in one frame use package "subfig"
\graphicspath{figures/}
\usepackage{pdfpages}		% To include a pdf file with
\usepackage{listings}
\usepackage[shortlabels]{enumitem}		% Modify the numeration counter: \begin{enumerate}[i.] \item \end{enumerate}
\usepackage{mathtools}		% Automatically loads "amsmath", "amstext" etc. automatically; [fleqn]: "Position equations at a fixed indent from the left margin rather than centered in the text column."
\usepackage{amssymb}	% Maths symbols like $\mathbb{N,Z,R}$ etc.
\usepackage{wasysym}	% TODO: read the docs ? used below for boxes
\usepackage{tikz}
%\usepackage[onehalfspacing]{setspace}

%% For testing:
\usepackage{lipsum}	    % Lorem ipsum dolor sit...


\def \workSheetNr {XX} %TODO

%% Setup of entities:
% Course specific entities:
\def \semester {Wi/SoSe 20xx}
\def \course {CourseXY}
\def \courseShort {cXY}
\def \group {Gruppe 42}

% Member specific entities:
\def \memberA {GroupMemberA}
\def \memberB {GroupMemberB}
\def \memberC {GroupMemberC}
%\def \memberD {GroupMemberD}
%\def \memberE {GroupMemberE}
%\def \memberF {GroupMemberF}
\def \memberAMartricle {0000000}
\def \memberBMartricle {0000001}
\def \memberCMartricle {0000002}
%\def \memberDMartricle {0000003}
%\def \memberEMartricle {0000004}
%\def \memberFMartricle {0000005}
\def \memberAMail {st000001@stud.uni-stuttgart.de}
\def \memberBMail {st000002@stud.uni-stuttgart.de}
\def \memberCMail {st000003@stud.uni-stuttgart.de}
%\def \memberDMail {st000003@stud.uni-stuttgart.de}
%\def \memberEMail {st000004@stud.uni-stuttgart.de}
%\def \memberFMail {st000005@stud.uni-stuttgart.de}

% Counting in $\mathbb{N}_0$!
\setcounter{page}{0}

%% PDF-internal properties	-	seems to make problems...
%\hyperrefsetup[
%	pdftitle = {\course~Üb \workSheetNr},
%	pdfauthor = {\memberA, \memberB, \memberC},	% \memberD, \memberE, \memberF
%	pdfsubject = {\course},
%	pdfkeywords = {\semester, \courseShort, \course, \group, \workSheetNr}
%	unicode
%]

%% Sets up header and cover sheet formatting.
\title{
	\Huge{Übungsblatt \workSheetNr} \\
	\Large{\course}
}
\author{ % TODO allign to the right edge with small margin
	\textbf{\memberA} [\memberAMartricle]~\textit{\memberAMail} \\
	\textbf{\memberB} [\memberBMartricle]~\textit{\memberBMail} \\
	\textbf{\memberC} [\memberCMartricle]~\textit{\memberCMail} \\
	%\textbf{\memberD} [\memberDMartricle]~\textit{\memberDMail} \\
	%\textbf{\memberE} [\memberEMartricle]~\textit{\memberEMail} \\
	%\textbf{\memberF} [\memberFMartricle]~\textit{\memberFMail} \\
}
\date{\today}

% Configuration of the header starts here
\pagestyle{fancy}
\fancyhf{}

\lhead{
	\semester \\
	\courseShort \\
	Übungsblatt \workSheetNr
}
\chead{\group}
\rhead{
	\memberA ~|~MtNr. \memberAMartricle \\
	\memberB ~|~MtNr. \memberBMartricle \\
	\memberC ~|~MtNr. \memberCMartricle %\\
	%\memberD ~|~MtNr. \memberDMartricle \\
	%\memberE ~|~MtNr. \memberEMartricle \\
	%\memberF ~|~MtNr. \memberFMartricle
}
\cfoot{\thepage}

%% Section for new custom commands or abbreviations, enhancing the overall look or "handling" while working on a emission
\newcommand{\exercise}[2]{\subsection*{Aufgabe {#1} {#2})}}  % Stole this one from Sandro haha! - Add \vspace{-2mm}? Add descriptor? $\triangleright$?


\begin{document}
	\maketitle
	\newpage % That's ok

	\exercise{1}{a}
	\lipsum


\end{document}

	% \textcolor{red}{text} to dye whole text blocks
	% \colorbox[red]{text in a box}\includepdf[pages={4,9}]{./pdf/filename.pdf} (who does that?)

%% Nice packages for useful symbols:
% To check your LaTeX ToDo-list: \check, {wasysymb} checked
% \usepackage{marvosym}		% To make use of the LIGHTNING symbol! / {stmaryrd} $\lightning$
% \usepackage{halloweenmath}	% If you want somebody to understand somethings not working: $\(big)skull$
% \usepackage{fontawesome5}	% To highlight something spicy \faPepperHot
% Explore this one even further :)

% Arrows and triangles: $\(R/r)ightarrow$ $\rightleftharpoons$ $\Leftrightarrow$ $\triangleright$ $\leadsto$ $\blacktriangleright$
% Box stuff: $\boxtimes$ $\boxempty$ {wasysym} \CheckedBox \Square \XBox (uff)
%\usepackage{upgreek}		% Nice looking Greek alphabet $\updelta$
% To want to swag your PDA up: \textdollaroldstyle
% For a AVL-tree rotation/ to display the state of thinking: $\circlearrowright$

%% Nice packages for stupid symbols:
% Just beautiful: $\ast$ $\bigstar$ $\bigvarstar$ $\divideontimes$ ({mathabx} $\blackdiamond$ $\Asterisk$ $\convolution$)
% If you want to crucify your equation: \dag
% If maths isn't scary enough: {halloweenmath} $\math(right/left/)ghost$ $\(big)skull$ $\math(right/left/)bat$ $\pumpkin$ $\(reverse/)mathcloud$
%\usepackage{keystroke}		% To display you are lost in space or you are a troll: \Ctrl\Alt\keystroke{F4}\Enter
% Signaling your tutor he/she should better stop reading: {marvosym} \Stopsign \Biohazard \Radioactivity \Laserbeam
% Express your feelings: {wasysym} \frownie \smiley \sun or just
%\usepackage{tikzsymbols} 	% \Cooley \Innocey \Cat \Annoey \Laughey \Neutrey \Sadey \Sey \Sleepey \Smiley \Tongey \Winkey \Xey
%\usepackage{recycle}		% When your solutions are even trash to you: \recycle \Recycle \RECYCLE - or just use \Recycling

%\usepackage{amsthm}							% If you want to flex on someone with \begin{lem}[(lemma name) Lemma stuff] \end{lemma}

%% Packages for (Pseudo-)Code:
%\usepackage[lined,algonl,boxed]{algorithm2e}
%\usepackage[]{algorithm}
%\usepackage{algpseudocode}

%% TiZ Libraries:
%\usetikzlibrary{automata,positioning}



%% Text Formatting:

% \textbf{text}			bold text
% \textit{text}			italics text
% \enquote{}				Does the "" thing the right/LaTeX way
% \quad \qquad			For line additional spacing

% \begin{enumerate}
% 	\item
% \end{enumerate}

% \begin{itemize}
% 	\item
% \end{itemize}


%% Figures (png, jpg, pdf,...) from external files:

% \begin{figure} [H]
%	\centering
% 	\includegraphics [width=1\textwidth]{./figures/figure.*}
% 	\caption{A figure showing something}
% 	\label{figureX}
% \end{figure}


%% Source code form external files:

% \lstinputlisting[language=java, firstline=37, lastline=45]{./files/codefile.*}


%TODO somehow places font in the middle
%% Source/ pseudo code within the document:

% \begin{lstlisting} [frame=single]
% 	for i:=maxInt to 0 do
% 	begin
% 		{do nothing}
% 	end;
% 	Write('Case insensitive ');
% 	Write('Pascal keywords.');
% \end{lstlisting}


%% Pseudo Code:

% \begin{algorithm}
% 	\caption{A Pseudo Code Programm}

% 	\begin{algorithmic} [1]
% 		\State Pick up a small pot
% 		\While{Ingredients not cut down}
% 			\State Cut Ingredients
% 		\EndWhile
% 		\If{Noodes are gritty and cook not italiano}
% 			\State Shower noodles with tap water  and mix trough
% 		\EndIf
% 		\State \Return plates with portions
% 	\end{algorithmic}
% \end{algorithm}


%% Simple Array:

% \begin{figure} [H]
% 	\centering

% 	\begin{tikzpicture}
% 		\coordinate (s) at (0,0);
% 		\foreach \number in {5,2,7,-5,16,12}{
% 			\node[minimum size=1cm, draw, rectangle] at (s) {\number};
% 			\coordinate (s) at ($(s) + (1,0)$);
% 		}
% 	\end{tikzpicture}

% 	\caption{An array}
% 	\label{arrayX}
% \end{figure}


%% Simple Matrix:

% \begin{figure} [H]
% 	\centering

% 	\[
% 	\begin{pmatrix}
% 		1 & 2 & 3 & 4 \\
% 		42 & 3 & 99 & 13
% 	\end{pmatrix}
% 	\]

% 	\caption{A Matrix}
% 	\label{matrixX}
% \end{figure}


%% "Simple" colored AVL Tree:

% \begin{figure} [H]
% 	\centering

% 	\begin{tikzpicture}[nodes={draw, circle}, level/.style ={sibling distance=60mm/#1}]
% 		\node {$55_{0}$}
% 			child {node {$34_{0}$}
% 				child {node [blue] {$21_{0}$}
% 					child [red] {node {$20_{0}$}}
% 					child [blue] {node {$22_{0}$}}
% 				}
% 				child {node {$37_{1}$}
% 					child {node {$35_{0}$}}
% 					child [missing]
% 				}
% 			}
% 			child {node {$70_{0}$}
% 				child {node {$60_{-1}$}
% 					child [missing]
% 					child [green] {node {$64_{0}$}}
% 				}
% 				child {node {$77_{0}$}
% 					child [missing]
% 					child {node {$80_{0}$}}
% 				}
% 			};
% 	\end{tikzpicture}

% 	\caption{A AVL-Tree with balancing}
% 	\label{tree_avlX}
% \end{figure}


%% (Not so) Simple (really messed up) 2-3-4 Tree:
%TODO There have to be better ways doing this...

% \begin{figure} [H]
% 	\centering

% 	\begin{tikzpicture}[nodes={draw,rectangle,rounded corners}, -]
% 		\node [blue] {51}
% 			child {node [blue] {26}
% 				child {node {13 16}
% 					child {node {7}}
% 					child {node {14 15}}
% 					child {node {19 }}
% 				}
% 				child [missing]
% 				child [missing]
% 				child {node {31}
% 					child {node {28 29}}
% 					child {node {44}}
% 				}
% 			}
% 			child [missing]
% 			child [missing]
% 			child [missing]
% 			child [missing]
% 			child [missing]
% 			child {node [blue] {73 87}
% 				child {node [blue] {62}
% 					child {node {56}}
% 					child {node {64 72}}
% 				}
% 				child [missing]
% 				child [missing]
% 				child [blue] {node {77 84}
% 					child [blue] {node {74 75}}
% 					child [green] {node {78 80}}
% 					child [blue] {node {86}}
% 				}
% 				child [missing]
% 				child [missing]
% 				child {node {94}
% 					child {node {91 92}}
% 					child {node {96 100}}
% 				}
% 			};
% 	\end{tikzpicture}

% 	\caption{A less nice 2-3-4 tree :/}
% 	\label{tree_2-3-4X}
% \end{figure}


%% Not really simple Read-Black-Tree:

% \begin{figure} [H]
% 	\centering

% 	\begin{tikzpicture}[nodes={draw, circle, fill=gray}, level/.style ={sibling distance=80mm/#1}]
% 		\node {55}
% 			child {node [fill = pink] {26}
% 				child {node {16}
% 					child {node {13}
% 						child {node [fill = pink] {7}}
% 						child {node [fill = pink] {14}}
% 					}
% 					child {node {19}
% 						child [missing]
% 						child {node [fill=pink] {24}}
% 					}
% 				}
% 				child {node {31}
% 					child {node {28}}
% 					child {node [fill = pink] {44}
% 						child {node {40}}
% 						child {node {46}}
% 						}
% 					}
% 				}
% 			child {node {70}
% 				child {node {59}
% 					child [missing]
% 					child {node [fill = pink] {62}}
% 				}
% 				child {node [fill = pink] {87}
% 					child {node {77}
% 						child [missing]
% 						child {node [fill = pink] {86}}
% 					}
% 					child {node {96}
% 						child {node [fill = pink] {95}}
% 						child {node [fill = pink] {97}}
% 					}
% 				}
% 			};
% 	\end{tikzpicture}

% 	\caption{A Red-Black-Tree}
% 	\label{tree_red-blackX}
% \end{figure}