\documentclass[12pt]{scrartcl}
\usepackage[ngerman]{babel}
\usepackage[utf8]{inputenc}
\usepackage[T1]{fontenc}
\usepackage{lmodern}	% May enhance latin letter quality
\usepackage[pdftex,hyperref,dvipsnames]{xcolor}		% Basic font coloring

%% Basic packages:
\usepackage{lipsum}	% For testing
\usepackage{graphicx}
\usepackage{listings}
\usepackage[inline]{enumitem}	% To be able to modify the numeration counter
\usepackage{amsmath,amssymb,amstext,amsthm}
%\usepackage{hyperref}
%\usepackage{csquotes}	% Provides advanced facilities for inline and display quotations

%% (Pseudo-)Code:
%\usepackage[lined,algonl,boxed]{algorithm2e}
%\usepackage[]{algorithm}
%\usepackage{algpseudocode}

%% Trees & Graphs:
%\usepackage{tikz}
%\usetikzlibrary{automata,positioning}

%% Refactored part handling the header and the overall page formatting specific stuff

\usepackage{fancyhdr}	% Package for header
\usepackage[a4paper,lmargin={1.5cm},rmargin={1.5cm},tmargin={3.5cm},bmargin = {2.5cm},headheight={1.5cm}]{geometry}	% page formatting - "headheight" change was necessary due to header
\usepackage{changepage}	% For adjust the left and right side margin width
\usepackage{float}	% Fixing the positioning of figures to section (with [H] option on the figure)

\usepackage{scrlayer}	%TODO: Use this package instead of geometry?

% Creating the header with entities from the ./CourseAndMemberEntities.tex file in the src directory:
\title{
	\Huge{\workSheet} \\
	\Large{\course}
}
\author{
	\textbf{\memberA} [\memberAMartikel ]~\textit{\memberAMail } \\
	\textbf{\memberB} [\memberBMartikel ]~\textit{\memberBMail } \\
	\textbf{\memberC} [\memberCMartikel ]~\textit{\memberCMail } \\ \\
}
\date{\today}

% Configuration of the header
\pagestyle{fancy}
\fancyhf{}

\lhead{
	\semester \\
	\courseShort \\
	\workSheet
}
\chead{
	\group
}
\rhead{
	\memberA ~|~\memberAMartikel \\
	\memberB ~|~\memberBMartikel \\
	\memberC ~|~\memberCMartikel
}
\cfoot{\thepage}
%% Defines all the necessary variable used in the main LaTeX file
%% Is meant to be edited at the beginning of the semester

% Course specific details:
\def \courseShort {DSA}
\def \course {Datenstrukturen \& Algorithmen}
\def \semester {Sommersemester 2022} %Wintersemester 2022
\def \group {Gruppe 06}

% Member specific details:
\def \memberA {Anton Sproll}
\def \memberB {Jannik Becker}
\def \memberC {Omid Pourazar}
\def \memberAMartikel {3592492}
\def \memberBMartikel {3597837}
\def \memberCMartikel {3608182}
\def \memberAMail {st178033@stud.uni-stuttgart.de}
\def \memberBMail {st177878@stud.uni-stuttgart.de}
\def \memberCMail {st178028@stud.uni-stuttgart.de}
%% Section for new custom commands or abbreviations, enhancing the overall look or "handling" while working on a emission


\newcommand{\exercise}[2]{\subsection*{Aufgabe #1, #2)} \vspace{-2mm}}  % Stole this one from Sandro haha!

%TODO
\def \workSheet {Übungsblatt XX}

\begin{document}
	\maketitle
	\newpage

	\exercise{1}{a}
	\begin{adjustwidth}{24pt}{}
		\lipsum
	\end{adjustwidth}

\end{document}

%% Text Formatting:

% \textbf{text}			bold text
% \textit{text}			italics text
% \enquote{}				Does the "" thing the right/LaTeX way
% \quad \qquad			For line additional spacing

% \begin{enumerate}
% 	\item
% \end{enumerate}

% \begin{itemize}
% 	\item
% \end{itemize}


%% Figures (png, jpg, pdf,...) from external files:

% \begin{figure} [H]
% 	\includegraphics [width=1\textwidth]{./figures/figure.*}
% 	\caption{Figure caption}
% 	\label{figureX}
% \end{figure}


%% Source code form external files:

% \lstinputlisting[language=java, firstline=37, lastline=45]{./files/codefile.*}


%% Source/ pseudo code within the document:

% \begin{lstlisting} [frame=single]
% 	for i:=maxInt to 0 do
% 	begin
% 		{do nothing}
% 	end;
% 	Write('Case insensitive ');
% 	Write('Pascal keywords.');
% \end{lstlisting}


%% Pseudo Code:

% \begin{algorithm}
% 	\caption{My Pseudo Code Programm}
% 	\begin{algorithmic} [1]
% 		\State Pick up a small pot
% 		\While{Ingredients not cut down}
% 			\State Cut Ingredients
% 		\EndWhile
% 		\If{Noodes are gritty and cook not italiano}
% 			\State Shower noodles with tap water  and mix trough
% 		\EndIf
% 		\State \Return plates with portions
% 	\end{algorithmic}
% \end{algorithm}


%% Simple Array:

% \begin{figure} [H]
% 	\centering
	
% 	\begin{tikzpicture}
% 		\coordinate (s) at (0,0);
% 		\foreach \number in {5,2,7,-5,16,12}{
% 			\node[minimum size=1cm, draw, rectangle] at (s) {\number};
% 			\coordinate (s) at ($(s) + (1,0)$);
% 		}
% 	\end{tikzpicture}

% 	\caption{A array}
% 	\label{arrayX}
% \end{figure}

% \begin{adjustwidth}{24pt}{}
% 	\lipsum
% \end{adjustwidth}


%% Simple Matrix:

% \begin{figure} [H]
% 	\centering
	
% 	\[
% 	\begin{pmatrix}
% 		1 & 2 & 3 & 4 \\
% 		42 & 3 & 99 & 13
% 	\end{pmatrix}
% 	\]

% 	\caption{A Matrix}
% 	\label{matrixX}
% \end{figure}


%% "Simple" colored AVL Tree:

% \begin{figure} [H]
% 	\centering
	
% 	\begin{tikzpicture}[nodes={draw, circle}, level/.style ={sibling distance=60mm/#1}]
% 		\node {$55_{0}$}
% 			child {node {$34_{0}$}
% 				child {node [blue] {$21_{0}$}
% 					child [red] {node {$20_{0}$}}
% 					child [blue] {node {$22_{0}$}}
% 				}
% 				child {node {$37_{1}$}
% 					child {node {$35_{0}$}}
% 					child [missing]
% 				}
% 			}
% 			child {node {$70_{0}$}
% 				child {node {$60_{-1}$}
% 					child [missing]
% 					child [green] {node {$64_{0}$}}
% 				}
% 				child {node {$77_{0}$}
% 					child [missing]
% 					child {node {$80_{0}$}}
% 				}
% 			};
% 	\end{tikzpicture}
	
% 	\caption{\textbf{Der $AVL_2$-Baum nach Einfügen der 64}}
% 	\label{tree-avl}
% \end{figure}


%% (Not so) Simple (really messed up) 2-3-4 Tree:
%TODO There have to be better ways doing this...

% \begin{figure} [H]
% 	\centering

% 	\begin{tikzpicture}[nodes={draw,rectangle,rounded corners}, -]
% 		\node [blue] {51}
% 			child {node [blue] {26}
% 				child {node {13 16}
% 					child {node {7}}
% 					child {node {14 15}}
% 					child {node {19 }}
% 				}
% 				child [missing]
% 				child [missing]
% 				child {node {31}
% 					child {node {28 29}}
% 					child {node {44}}
% 				}
% 			}
% 			child [missing]
% 			child [missing]
% 			child [missing]
% 			child [missing]
% 			child [missing]
% 			child {node [blue] {73 87}
% 				child {node [blue] {62}
% 					child {node {56}}
% 					child {node {64 72}}
% 				}
% 				child [missing]
% 				child [missing]
% 				child [blue] {node {77 84}
% 					child [blue] {node {74 75}}
% 					child [green] {node {78 80}}
% 					child [blue] {node {86}}
% 				}
% 				child [missing]
% 				child [missing]
% 				child {node {94}
% 					child {node {91 92}}
% 					child {node {96 100}}
% 				}
% 			};
% 	\end{tikzpicture}

% 	\caption{\textbf{A less nice 2-3-4 tree :/}}
% 	\label{tree_2-3-4}
% \end{figure}


%% Not really simple Read-Black-Tree:

% \begin{figure} [H]
% 	\centering

% 	\begin{tikzpicture}[nodes={draw, circle, fill=gray}, level/.style ={sibling distance=80mm/#1}]
% 		\node {55}
% 			child {node [fill = pink] {26}
% 				child {node {16}
% 					child {node {13}
% 						child {node [fill = pink] {7}}
% 						child {node [fill = pink] {14}}
% 					}
% 					child {node {19}
% 						child [missing]
% 						child {node [fill=pink] {24}}
% 					}
% 				}
% 				child {node {31}
% 					child {node {28}}
% 					child {node [fill = pink] {44}
% 						child {node {40}}
% 						child {node {46}}
% 						}
% 					}
% 				}
% 			child {node {70}
% 				child {node {59}
% 					child [missing]
% 					child {node [fill = pink] {62}}
% 				}
% 				child {node [fill = pink] {87}
% 					child {node {77}
% 						child [missing]
% 						child {node [fill = pink] {86}}
% 					}
% 					child {node {96}
% 						child {node [fill = pink] {95}}
% 						child {node [fill = pink] {97}}
% 					}
% 				}
% 			};
% 	\end{tikzpicture}

% 	\caption{\textbf{A nice Red-Black-Tree}}
% 	\label{tree-red-black}
% \end{figure}